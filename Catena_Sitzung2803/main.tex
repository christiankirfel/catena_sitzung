\documentclass[a4paper, 11pt]{article}
\usepackage{comment} % enables the use of multi-line comments (\ifx \fi) 
\usepackage{fullpage} % changes the margin
\usepackage[german]{babel}
\usepackage[utf8]{inputenc}
\usepackage{graphicx}
\usepackage{multicol}
\usepackage{float}
\usepackage{fancyhdr}
\usepackage{enumitem}
\pagestyle{fancy} 
\usepackage{pdfpages}
%\usepackage[head=128pt]{geometry}
\title{Sitzungs-protokoll}
\author{Christian Kirfel}
\usepackage{geometry}
\setlength{\footskip}{0.1pt}
\setlength{\headheight}{80pt}
\setlength{\topmargin}{0pt}
\setlength\parindent{0pt}
\fancypagestyle{style1}{
\rhead{\includegraphics[width=4cm]{logo_catena}}
\cfoot{
\makebox{}\\
\makebox{}\\
\hspace{15cm}
\makebox[0.1\linewidth]{\rule{0.1\linewidth}{0.1pt}} \hspace{1cm} \makebox[0.1\linewidth]{\rule{0.1\linewidth}{0.1pt}} \hspace{1cm} \makebox[0.1\linewidth]{\rule{0.1\linewidth}{0.1pt}} \hspace{1cm} \makebox[0.1\linewidth]{\rule{0.1\linewidth}{0.1pt}} \hspace{1cm}\\}
}
\begin{document}
\pagestyle{style1}

\textbf{Datum: 28.03.2020} % inserir data aqui

\textbf{Ort: Remote, Catena Discord} % Definir local da reunião 

\textbf{Teilnehmer:} %
\begin{description}
\item Martin Reinicke, Vorsitzender
\item Dr.~Michael Schumacher, stellv. Vorsitzender
\item Volker Nießen, Geschäftsführer
\item Christian Kirfel, Schriftführer
\item Jörg Zwitters, Verbindungslehrer
\item Dominik Hoven, Beisitzer
\item Max Frauenrath, Beisitzer
\end{description}

\makebox[\linewidth]{\rule{\linewidth}{0.4pt}}\\
\textbf{Programmpunkte:} 
\begin{enumerate}
\item Neues aus der Schule
\item Aktuelle Entwicklungen aus der Stiftung
\item Jahreshauptversammlung und Schulfest
\item Kasse
\item Digitalprojekte
\end{enumerate}
\makebox[\linewidth]{\rule{\linewidth}{0.4pt}}\\

\newpage

\section*{1. Neues aus der Schule}


Für Schulen wurde eine vorerst zweiwöchige Quarantäne verhängt, die vermutlich ein vorerst andauernder Zustand sein wird.
E-learning findet mit Ipads und Itunes U statt. Das Feedback von Schülern und Eltern ist vornehmlich positiv.

Abiturprüfungen wurden verschoben und Ausflüge sowie Exkursionen vorerst ausgesetzt.

Über ein Stattfinden eines Segelkurses muss nachgedacht werden:

\begin{itemize}
\item Bietet Günther den Kurs an?
\item Einladungen müssten nun versandt werden
\item Eltern sind finanziell vorsichtig
\item Kurzfristigkeit: aktuell sind am See durch die Stadt Heimbach alle Veranstaltungen gesperrt, Vorlaufzeit ist durch die Eltern bestimmt, prinzipiell sind wenige Tage möglich
\item Elternschreiben: den Eltern bei Interesse eine kurzfristige Teilnahme vorschlagen sobald der Unterricht wieder einsetzt
\item Für Reparaturen müssten einige Gelder eingeplant werden
\end{itemize}
\textbf{Antrag}: Geld ausgeben um die Amadeus zu sanieren: Keine Gegenstimmungen, keine Enthaltungen
\emph{Antrag angenommen}
Digitalpakt: Weitere Investitionen sollen kommen, schnelle Glasfaserleitungen, Beamererneuerung von Herrn Frauenkron abgelehnt.
Im Allgemeinen wird das Paket ausgiebig in Anspruch genommen.

\section*{2. Aktuelle Entwicklungen aus der Stiftung}

Helmut Lanio kann krankheitsbedingt nicht mehr als Vorstands-Vorsitzender agieren.
Neuorganisation und Strukturierung des Vorstands stehen an.
Demnach muss über einen neuen Catena Vorstand nachgedacht werden, bevorzugt aus der jüngeren Generation.

Bänke und Möbel für die Schule sind organisiert. Die Bänke haben bei einem deutschen Produzenten nach Bestellung eine Lieferzeit von 5 Wochen. Die italienischen Möbel, der aktuellen Lage geschuldet, eine unbekannte Lieferzeit.

\section*{3.Jahreshauptversammlung und Schulfest}

Das Stattfinden von sowohl Schulfest als auch Hauptversammlung hängt vollständig von der Entwicklung der aktuellen Corona-Epidemie ab.
Im besten aber unwahrscheinlichen Fall kann beides normal stattfinden. Wahrscheinlich ist aber, dass das Schulfest aufgrund mangelnder Vorbereitungszeit ausfällt und eine Hauptversammmlung eventuell digital stattfinden muss.

Ein Catena-Fest, wenn nötig minimalistisch, in Kombination mit dem Jubiläum der Jahrgänge sollte wenn irgendwie möglich auch kurzfristig geplant werden. Einladungen sollte wenigstens zwei Wochen vorher rausgehen.
Für ein Schulfest müsste noch vor den Ferien ein Termin zugesagt werden.
Für ein Schulfest müssten Räumlichkeiten der Schule zur Verfügung stehen. Alternativ müsste sich mit dem Hotel in Verbindung gesetzt werden.

Es wurden zwei Prüfpunkte festgelegt:

\textbf{Prüfpunkt Schulfest/Catena}: 30.05.

\textbf{Prüfpunkt Catena}: 27.06.
  
  
Die Hauptversammlung könnte auch mit dem Tag der offenen Tür verbunden werden.
 
\section*{4. Kasse}

24000 VR

Jahr 2019 so weit wie möglich aufgearbeitet.

Schatzmeister würde machen: Friedhelm Schmitt

Ermächtigung: Unklar: Kann bis zur nächsten Sitzung kommissarisch berufen werden werden.
Kann als vorläufiger Schatzmeister die Arbeit übernehmen.
Mahnen wurde aus Gründen der finanziellen Unsicherheit zurückgestellt.


\textbf{Mitgliederentwicklung:}

35 Austritte, 10 neue Mitglieder

380 Mitglieder insgesamt

Wie läuft Werbung der Neumitglieder mit dem 1 Euro Beitrag?
Werbung sollte mit der Zeugnisausgabe stattfinden.
Die SV könnte hier eine Werberolle übernehmen, Thema mit Motivation Bänke in die Wege leiten.

Thermobecher sind noch vorhanden.

\section*{5. Digitalprojekte}

\subsection*{Dropin}

Wir befragen Bekannte, die bereit wären, im Rahmen der Quarantäne auf dem Discord Server interessierten Schülern Fragen zu beantworten. Dazu besprechen wir uns am 29.03 erneut.
So könnte auch die Bekanntheit des Servers erhöht werden.

Sobald aktiv sollte dies auf facebook gesetzt werden.

\subsection*{Discord}

Man kann beginnen SchülerInnen einzuladen und den Discord langsam publik zu machen

Auch nicht Catena Mitgliedern wird der Zugriff erlaubt.

\subsection*{Website}

Keine Zeit Dinge zu schreiben. Informationen zum Hochladen müssten Dominik direkt weitergegeben werden.

Möglicherweise können Inhalte von der Schule übernommen werden.

Andere Social Media Präsenzen sollten auf der website aktualisiert werden.

Weitere Verantwortliche für die Website werden hinzugefügt, um die Arbeitslast aufzuteilen.

\subsection{LinkedIn und Xing}

Gruppe wurde erstellt, Mitglieder sollten hinzugefügt werden.
Weitere Themen sollten kommuniziert werden.

Teilnehmer sollten angehalten sein auch offiziell dem Verein beizutreten





\end{document}
