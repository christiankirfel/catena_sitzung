\documentclass[a4paper, 11pt]{article}
\usepackage{comment} % enables the use of multi-line comments (\ifx \fi) 
\usepackage{fullpage} % changes the margin
\usepackage[german]{babel}
\usepackage[utf8]{inputenc}
\usepackage{graphicx}
\usepackage{multicol}
\usepackage{float}
\usepackage{fancyhdr}
\usepackage{enumitem}
\pagestyle{fancy} 
\usepackage{pdfpages}
%\usepackage[head=128pt]{geometry}
\title{Sitzungs-protokoll}
\author{Christian Kirfel}
\usepackage[margin=1in]{geometry}
\setlength{\footskip}{0.1pt}
\setlength{\headheight}{80pt}
\setlength{\topmargin}{0pt}
\setlength\parindent{0pt}
\fancypagestyle{style1}{
\rhead{\includegraphics[width=4cm]{logo_catena}}
\cfoot{
\makebox{}\\
\makebox{}\\
\hspace{15cm}
\makebox[0.1\linewidth]{\rule{0.1\linewidth}{0.1pt}} \hspace{1cm} \makebox[0.1\linewidth]{\rule{0.1\linewidth}{0.1pt}} \hspace{1cm} \makebox[0.1\linewidth]{\rule{0.1\linewidth}{0.1pt}} \hspace{1cm} \makebox[0.1\linewidth]{\rule{0.1\linewidth}{0.1pt}} \hspace{1cm}\\}
}

\newcommand\signature[2]{% Name; Department
\noindent\begin{minipage}{5cm}
    \noindent\vspace{3cm}\par
    \noindent\rule{5cm}{1pt}\par
    \noindent\textbf{#1}\par
    \noindent#2%
\end{minipage}}



\begin{document}
\pagestyle{style1}

\textbf{Datum: 22.08.2021} % inserir data aqui
\textbf{Sitzungszeit: 19:00 - 00:00}
\textbf{Ort: Remote, Zoom} % Definir local da reunião 

\textbf{Teilnehmer:} %
\begin{description}
\item Yannick Hansen, Vorsitzender
\item Dr.~Michael Schumacher, stellv. Vorsitzender
\item Volker Nießen, Geschäftsführer
\item Friedhelm Schmitt, Schatzmeister
\item Christian Kirfel, Schriftführer
\item Dominik Hoven, Beisitzer
\item Martin Reinicke, Beisitzer
\item Phlipp Lanio, Beisitzer
\end{description}

\makebox[\linewidth]{\rule{\linewidth}{0.4pt}}\\
\textbf{Programmpunkte:} 
\begin{enumerate}
\item Digitale Veranstaltung
\item Mitglieder-Hauptversammlung
\item Integration der alten Platform
\end{enumerate}
\makebox[\linewidth]{\rule{\linewidth}{0.4pt}}\\

\newpage

\section*{Digitale Veranstaltung}

Die Möglichkeit einer 40-Jahr Feier bleibt ungewiss.
Martin schlägt ein Konzept eines virtuellen Meetings vor.
Meeting für 2-3 Stunden, Moderatoren in Steinfeld vor Ort.
Einspieler und Einladungen einyelner Personen ist möglich.
Einladung mit erfolderlicher Anmeldung erfolgt per Email.
Der gesamte Vorstand sollte anwesend sein.
Die Mitglieder der Catena sollten sich in den Vorstellungen wiederfinden.
Gründungsmitglieder, Projekte und so weiter sind mögliche Themen.
Martin schlägt vor, dass man einige Personen zu Inteviews einlädt.
Als Überraschung könnte man eine Flasche Klosterbier und ein Glas versenden.

Christian schlägt ein eule Kamerasystem vor.
Die Frage ist ob man die Hauptversammlung mit der digitalen Veranstaltung verbindet.
Bevorzugt ist die HV vor der Veranstaltung zu planen.

Ein Probelauf für die Technik wäre notwendig.
Gegeben den notwendigrn Vorlauf sollte ein Termin im Oktober geufnden werden.
Vorläufig wird der 30. Oktober festgehalten.

Martin Reinicke übernimmt die Projektleitung.

Christian schlägt ein Quiz vor, dass man in digitalen Räumen spielt.

Für Videoeinspieler ist ein Rundgang durch die Schule angedacht.
Schüler, Kloster und neue Räume der Schule sind Vorschläge.

Lobby im Hotel oder Lehrer Lounge sind gute Räume.

Yannick schlägt einen virtuellen Rundgang durch die Schule vor. Hier könnte ma die FilmßAG fragen.

HJK Seite überprüfen, ob Filmmaterial besteht.

Anmeldung sollte so flexibel wie möglich sein.

Martin übernimmt die Moderation.

Die Tageszeit sollte in den späteren Nachmittag verschoben werden.
Allerdings müssen die Lichtverhältnisse im Oktober bedacht werden.

Wer was macht, muss besprochen werden. Sie Planung sollte sofort beginnen.

Zimmer sollte organisiert werden.

Boot liegt auf Eis.

\section*{Mitglieder-Hauptversamlung}

Zur Hauptversamlung steht die Wahl der Vorstands an.
Die HV soll vor dem digitalen Event stattfinden.

\section*{Integration der alten Platform}

\section*{Sonstiges}




\newpage

Protokoll: Christian Kirfel


%\signature{Yannick Hansen}{Vorsitzender}\hfill\signature{Christian Kirfel}{Schriftführer}


\end{document}
