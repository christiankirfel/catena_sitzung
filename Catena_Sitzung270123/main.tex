\documentclass[a4paper, 11pt]{article}
\usepackage{comment} % enables the use of multi-line comments (\ifx \fi) 
\usepackage{fullpage} % changes the margin
\usepackage[german]{babel}
\usepackage[utf8]{inputenc}
\usepackage{graphicx}
\usepackage{multicol}
\usepackage{float}
\usepackage{fancyhdr}
\usepackage{enumitem}
\pagestyle{fancy} 
\usepackage{pdfpages}
%\usepackage[head=128pt]{geometry}
\title{Sitzungs-protokoll}
\author{Christian Kirfel}
\usepackage[margin=1in]{geometry}
\setlength{\footskip}{0.1pt}
\setlength{\headheight}{80pt}
\setlength{\topmargin}{0pt}
\setlength\parindent{0pt}
\fancypagestyle{style1}{
\rhead{\includegraphics[width=4cm]{logo_catena}}
\cfoot{
\makebox{}\\
\makebox{}\\
\hspace{15cm}
\makebox[0.1\linewidth]{\rule{0.1\linewidth}{0.1pt}} \hspace{1cm} \makebox[0.1\linewidth]{\rule{0.1\linewidth}{0.1pt}} \hspace{1cm} \makebox[0.1\linewidth]{\rule{0.1\linewidth}{0.1pt}} \hspace{1cm} \makebox[0.1\linewidth]{\rule{0.1\linewidth}{0.1pt}} \hspace{1cm}\\}
}

\newcommand\signature[2]{% Name; Department
\noindent\begin{minipage}{5cm}
    \noindent\vspace{3cm}\par
    \noindent\rule{5cm}{1pt}\par
    \noindent\textbf{#1}\par
    \noindent#2%
\end{minipage}}



\begin{document}
\pagestyle{style1}

\textbf{Datum: 27.01.23} % inserir data aqui
\textbf{Sitzungszeit: 18:00 - 20:00}
\textbf{Ort: Remote, Zoom} % Definir local da reunião 

\textbf{Teilnehmer:} %
\begin{description}
\item Yannick Hansen, Vorsitzender
\item Kimberly Sarlette, stellv. Vorsitzender
\item Sebastian Heinen, Schatzmeister
\item Christian Kirfel, Schriftführer
\item Dominik Hoven, Beisitzer
\item Phlipp Lanio, Beisitzer
\item Martin, Beisitzer
\end{description}

\newpage

\makebox[\linewidth]{\rule{\linewidth}{0.4pt}}\\
\textbf{Programmpunkte:} 
\begin{enumerate}
\item Begrüßung
\item Catena Umfrage
\item Logo
\item Image Film
\item Catena Tag
\item Catena Büro
\item Catena Fahrt
\item SV Seminar
\item Rabatte
\item Abiball
\item Segelkurs
\item 100 Jahre HJK
\item Wegweiser
\item Sonstiges
\end{enumerate}
\makebox[\linewidth]{\rule{\linewidth}{0.4pt}}\\

\newpage

\section*{Umfrage}

Sebastian hat eine Umfrage für bestehende und potentielle Mitglieder vorgeschlagen.

\section*{Catena Logo}

Yannick schlägt vor den Schriftzug um Freunde zu erweitern.
Sebastian verweist darauf, dass es vorrangig ein Verein für Ehemalige ist.
Martin zeigt auf, dass die Ordnung nicht ausschließlich auf Ehemalige bezogen ist.
Martin schlägt vor, es für ein Jahr zu testen.

Max schlägt Änderung von am zu des vor. Keiner ist dagegen.
Christian ist gegen einen gegenderten Begriff und bevorzugt dann die rein weibliche Form
Yannick fragt Sabine nach einem Designvorschlag.

\section*{Image Film}

Max findet ein Video gut. Es ist aber ein anspruchsvolles Konzept
Es ist gut, dass die Schüler es machen.
Christian schlägt vor es nicht zu übereilen sondern bei den geplanten Aktivitäten Filmmaterial zu sammeln.

\section*{Catena Tag}

Kim fügt hinzu, dass wir alles mit Sport aussortieren sollen, da sich alle zum Fußballspielen anmelden.
Der Catena Tag soll an einem Freiatg sein.
Hier sollen die potentiellen neuen Mitglieder geködert werden.

Martin schlägt ein Zoom meeting in Folge vor. Hier soll der Tag zusammengefasst werden.
Da könnte man Christians Vortrag bewerben.
Feedback ist gut.

Interview Training schlägt Martin als Alternative vor zu Umweltschutz

Martin fragt außerdem ob man Austauschprogramme thematisiert
Yannick und Kim haben das schon im Berufsvorbereitungstag eingebunden

SoWi, Sebastian ist nicht sicher ob es mehrere Themen sind.

Martin wirft die Frage auf wie lange der Tag dauert und wie viele Teilnehmer die Aktivitäten tragen können.
Die Q2 hat etwa 80 Schüler. Die Zahl der Parallelveranstaltungen soll geplant werden.

Weitere Themen wie Bildbearbeitung oder Kameratechnik sind vorgeschlagen.

Yannick fragtb wer Projektleiter wird.
Keiner will.

Yannick ist für eine Aufwandtsentschädigung .
Er fragt Martin nach seine Expertise.
Martin sagt Fahrtkosten geht immer. Für weitere Löhne gibt es Diskussion

Sebastian sagt, 50 Euro Erfrischungsgeld ist bei Wahlhelfern typisch.
Keiner ist gegen eine solche Entschädigung.

\section*{Catena Büro}

Das Büro würde für 6 Menschen passen
Meistens sollte das reichen, sonst können Seminarräume genutzt werden.

Abstimmung über Notwendigkeit und Budget:

Kim schlägt vor, dass man nicht zwingend Beamer braucht wenn man immer in den Seminarraum kann.
Martin findet es angemessen.
Yannick schlägt Maximalbudget von 2000 Euro vor. Martin findet 2000 bis 3000 ok.
Alle sind dafür. Keiner ist dagegen.

Martin meint Yannick soll Christoph Böhnke fragen bevor er die Möbel entsorgt.
Martin empfiehlt Teppichboden wegen des Schalls.

Der 22.4. wird vorgeschlagen um die Renovierung durchzuführen.
Von Samstag auf Sonntag, Arbeit mit anschließendem Essen.
Start um ist um 9 Uhr. Kim kann dann nicht ausschlafen.

Yannick muss das Essen und Zimmer reservieren.

Max will zum Ikea Möbel besorgen.

\section*{Catena Fahrt}

Dominik wirft ein, dass der Eindruck entsteht, dass man Nichtmitglieder unterstützt.
München ist das realistischste Ziel aber auch Krakau wird vorgeschlagen.

Fraglich ist Größe und Art der Bezuschussung.

Martin schlägt vor für München erst einmal die Eckdaten einzuholen und sich ein Bild zu machen.

\section*{SV Seminar}

Martin bringt ein, dass die Fahrt keine Spaßfahrt sondern vielmehr notwendige Planung der SV mit Spaßfaktor ist.

\section*{Rabatte für Mitglieder}

Sebastian erwartet ein Ausnutzen des Schulfestrabattes.
Martin findet das gar nicht schlimm und geht von einer guten Werbemaßnahme aus.
Martin würde den Rabatt auf der Preistafel gut sichtbar machen.
Sebastian weist in seiner Funktion auf das finanzielle Risiko hin.

Mitgliedsausweis sollten zugeschickt werden. Bei fehlender Adresse kann per email erfragt oder allgemein Bestätigung eingeholt werden.
Schwierigkeiten bei Email verschicken

\section*{Abiball}

-

\section*{Segelkurs}

Philip berichtet. Die stark begrenzten Plätze sind der einschränkende Faktor.
Es muss mehr Information eingeholt werden bevor wir Mitgliedern eine gute Möglichkeit für einen Segelkurs vorschlagen können.

\section*{100 Jahre HJK}

Martin ist für das Outdoor Klassenzimmer.

Kim möchte im Rahmen einer Feier einen Vortrag vorschlagen. Das Thema soll die Erinnerung an die Schule aus verschiedenen Generationen sein.


\section*{Wegweiser für die Zukunft}

Beim letzten Termin wurde leider für die falsche Stufe vorgeschlagen.

Für den nächsten Termin sucht Yannick einen Vortragspartner.
Max weiß noch nicht, ob er dann im Urlaub ist.

\section*{Verschiedenes}

Martin verweist darauf, dass die Verbundenheit mit der Schule nicht mehr in der Satzung verankert ist.
In der nächsten HV sollte der Zweck des Vereins angesprochen werden.

Laut Martin suchen die Rotarier (Rotary Orden?) noch Bewerber für ein vollfinanziertes Auslandsjahr. Das wäre während der Schulzeit.
Das Schuljahr muss in der Regel wiederholt werden.
Laut Dominik gab es einen Nachmittag ohne Schule gibt. Das würde sich für eine Informationsveranstaltung eignen.

Christian schlägt vor ein Treffen in Person im Raum Köln/Bonn zu machen.
Im Rahmen dessen könnte man Martins Interviewtraining machen.
Martin wäre dazu bereit. Christian würde eine Interessentengruppe finden und dann in Absprache mit Martin Ort und Zeit finden.

\newpage

Protokoll: Christian Kirfel


\signature{Yannick Hansen}{Vorsitzender}\hfill\signature{Christian Kirfel}{Schriftführer}

\end{document}
