\documentclass[a4paper, 11pt]{article}
\usepackage{comment} % enables the use of multi-line comments (\ifx \fi) 
\usepackage{fullpage} % changes the margin
\usepackage[german]{babel}
\usepackage[utf8]{inputenc}
\usepackage{graphicx}
\usepackage{multicol}
\usepackage{float}
\usepackage{fancyhdr}
\usepackage{enumitem}
\pagestyle{fancy} 
\usepackage{pdfpages}
%\usepackage[head=128pt]{geometry}
\title{Sitzungs-protokoll}
\author{Christian Kirfel}
\usepackage[margin=1in]{geometry}
\setlength{\footskip}{0.1pt}
\setlength{\headheight}{80pt}
\setlength{\topmargin}{0pt}
\setlength\parindent{0pt}
\fancypagestyle{style1}{
\rhead{\includegraphics[width=4cm]{logo_catena}}
\cfoot{
\makebox{}\\
\makebox{}\\
\hspace{15cm}
\makebox[0.1\linewidth]{\rule{0.1\linewidth}{0.1pt}} \hspace{1cm} \makebox[0.1\linewidth]{\rule{0.1\linewidth}{0.1pt}} \hspace{1cm} \makebox[0.1\linewidth]{\rule{0.1\linewidth}{0.1pt}} \hspace{1cm} \makebox[0.1\linewidth]{\rule{0.1\linewidth}{0.1pt}} \hspace{1cm}\\}
}

\newcommand\signature[2]{% Name; Department
\noindent\begin{minipage}{5cm}
    \noindent\vspace{3cm}\par
    \noindent\rule{5cm}{1pt}\par
    \noindent\textbf{#1}\par
    \noindent#2%
\end{minipage}}



\begin{document}
\pagestyle{style1}

\textbf{Datum: 06.11.2021} % inserir data aqui
\textbf{Sitzungszeit: 13:00 - 14:00}
\textbf{Ort: Remote, Zoom} % Definir local da reunião 

\textbf{Teilnehmer:} %
\begin{description}
\item Yannick Hansen, Vorsitzender
\item Dr.~Michael Schumacher, stellv. Vorsitzender
\item Volker Nießen, Geschäftsführer
\item Friedhelm Schmitt, Schatzmeister
\item Christian Kirfel, Schriftführer
\item Jörg Zwitters, Verbindungslehrer/Beisitzer
\item Leonhard Paulig, Kassenprüfer
\item Sebastian Heinen, Kassenprüfer
\item Dominik Hoven, Beisitzer
\item Martin Reinicke, Beisitzer
\item Max Frauenrath, Beisitzer
\item Philip Lanio, Beisitzer
\item Thomas Frauenkron, Schulleiter
\item Dirk Udo Fricke, Mitglied
\item Günter M. Lutsch, Mitglied
\item Alfred Kahl
\item Anja Pick
\item Burkhard Cremer
\item J. Boshold
\item Lothar Esser
\item Martin Heinen
\item Hermann Preußner
\item Rudolf Gerhards
\item Stefan Mathey
\item Thomas Webr
\item Wolfgang Bergsdorf
\item Wolfgang Schmitz
\item Michael Schumacher
\item Clara Keutgen
\item Ben
\item Ute Stolz
\end{description}

\makebox[\linewidth]{\rule{\linewidth}{0.4pt}}\\
\textbf{Programmpunkte:} 
\begin{enumerate}
\item Begrüßung
\item Feststellung der Beschlussfähigkeit
\item Genehmigung des Protokolls der MV September 2020
\item Bericht des Vorstandes
\item Bericht des Schatzmeisters
\item Bericht des Kassenprüfers
\item Entlastung des Vorstandes
\item Wahl des Vorsitzenden, Wahl des Schatzmeisters, Wahl eines Kassenprüfers
\item Persönliche Anträge
\item Verschiedenes
\end{enumerate}
\makebox[\linewidth]{\rule{\linewidth}{0.4pt}}\\

\newpage

\section*{Begrüßung}

Die Hauptversammlung findet wegen Covid 19 hybrid statt. Der analoge Teil sitzt in Steinfleld.
Yannick erinnert an die im Anschluss stattfindende Veranstaltung statt.

\section*{Beschlussfähigkeit}

Für Kommentare muss über Zoom regiert werden.

Die Beschlussfähigkeit wird auf Antrag des Vorsitzenden festgestellt.

\section*{Genehmigung des Protokolls der MV September 2019}

Das Protokoll wird ohne Widerspruch genehmigt.

Bericht Steinfleld

Corona hat auch Steinfeld und das schulleben im Griff.
Es gab einiges an digitalem Unterricht und einen Lockdown.
Durch die Digitalisierung war der Unterrich von zuhause sehr erfolgreich und es wurde von der Digitalisierung profitiert.
Durch den Digitalpakt ist noch mehr Geld in die Schule geflosse. Zwischen den access points wurde Glasfaser gelegt.
Der erste ipad Jahrgang hat Abiur gemacht.
Die Sportstätten auf dem Außgelenede wurden ausgebaut. Inklusive Laufbahn, Kuglstossen und Weitsprung.
Die Segel AG konnte nicht regelmäßig stattfinden und die Amadeus steht seit zwei Jahren in der Scheune.
Bald kann das Projekt hoffentlich wieder anlaufen.

Bericht Martin Stiftung

Digitale Weiterentwiclung. Website und Kapitalerhalt sollten erneuert werden.
Die Stifutng sollte bald in der Lage sein die Schule langfristig zu unterstützen. 
Später e´mehr in der Veranstaltung

Schülervertretung

Kim übernimmt die Kommunikation SV Catena
Eine neue SV wurde gewählt. 10 Schüler auf einem Seminar
Die nehmen an Seminaren teil, die sie mit Ideen und Grundlagen einer Schülervertretung unterrichtet.
Teil der SV nimmt an der Feier teil.
Die ehemalige SV hat einen Zukunftstag organisiert.

Berufsberatung

Dominik ist der Ansprechpartner
Nach den weihnachtsferien sol es geneaurere Vorberetinungen zu Studum, Work and travel und Ausbildung gehene,
Ein besonderer Fokus liegt auf den alltäglichen Dingen von Geldkosten bis Sportveranstaltungen.
Yannikc, Kim, Martin und Mx haben besprochen was die wichtigen Infos sind die uns selber gefehlt haben.

Yannick fügt hinzu dass die Resonanz seitens der Schule sehr positiv war

Christian Kurse und Vorträge

Yannick fügt hinzu dass wir die Catena Mitgleider mehr einbeziehen swollen.

Segel AG

Gleich im Videoeinspieler


\section*{Bericht des Vorstandes}

Neue Website, wir wollten ein neueres moderneres Design.
Yanick hat die website selber rpogrammiert und betreibt die website.
Hier haben wir mehr Automatisierung und ein neues Newsletter programm.
Auf der websie gibt es zusätzlich neue Blogeinträge.

Die website bietet auch eine bessere Kommunikation von Veranstaltungen mit direkter automatisierter Anmedlung

Bessere Mitgliederverwaltung durch neue Software. Hier wird mehr getreamlined
Hier kann die Arbeit durch Dezentralisierung und eine inline plattform gleichmäßig auf mehr Mitgleider verteilt.
Volker kann sich auf die Mitgleider und der Schatzmeister auf die Finanzen konzentrieren.

Die Stammdaten im Verein können nun digital von den Mitgliedern selber kontrolliert werden.

ein neues Newsletter programm ist rapidmail. Es handelt sich ebenfalls um ein Baukastensystem, dass das einfache Versenden schöner newsletter erlaub
Diese werden direkt in das Mitgliedersystem eingespeist

Außerdem hat die Catena die schule beim Bau einer Calisthenics Anlage unterstützt.
Diese wurden unter anderrm vo den Sportlehrern aufgebaut.

Außerdem trägt der Vorstand bereits Polos die bald für alle geplant sind.

Ausblick

Wir wollen präsenter für die Schüler sein. Vor allem in Hibnlick auf berusberatung
Nächstes Jahr hoffen wir wieder in Person in Stifneld mit dem Mitlgiedern feiern zu können.
Ein Sv Ausglug ist auch erneut geplant.



\subsection*{Kloster und Stiftung}



\subsection*{Schule}





\subsection*{Kloster/Kuratorium}



\section*{Bericht des Schatzmeisters}

Kassenbericht des letzten Jahres, wird in der Präentation gezeigt und hier übernommen.
Überschuss von 13000 irgendwas
31.12.20 18000
Eine Folie wird zusätzlich ins Protokoll genommen.

Yannick fügt hinzu dass alle Infos auch ind en Bericht übernommen werden

\section*{Bericht des Kassenprüfers}
Leonhard Paulig hat den Bericht vershickt.
Er hat mit Sebastian geprügt
Corna bedingt wurde viel gespart aber die Finanzen wurden sehr ordentlich geführt.


\section*{Mitgliederentwicklung}

Hier wird die Folie übernommen.

\section*{Entlastung des Vorstandes}

Es gibt keine Gegenstimmen zur Entlastung des Vorstandes. Der Vorstand enthält sich. Damit wird der Vorstand mit Bestätigung des Protokolls entlastet.

Leonahrd Paulig macht die Neuwahl.
Keine Gegenstimmen für Yannick Hansen

Neuwahl des weiteren Vorstandes.
Der stellvertret´dende Vostizende tritt zurück.
Kim wird vorgeschlagen. Gibt es weitere Vorschläge.
Es gibt keine Gegenvorschläge
es gibt keine Gegenstimmen.
Es gibt keine Enthaltugnen.
Kim nimmt die Waahl an.
Geschäftsführer Voker Nießen, geschäftsführer
Keine Gegenvorschläge
Keine Gegenstimmen
Voker nimmt die Wahl an
Schatzmeister Friedhelm Schmitt
Neuer Kandidat Sebastian heinen
ssebastian stellt sich kurz vor
Keine Gegenvorschläge
Keine Gegenstimmen
Sebastian nimmt die wahl an
Schriftführer Christian Kirfel
Keine Gegenvorschläge
Keine Gegenstimmen
Keine Enthaltungen
Alle nehmen die wahl an.
Alle Beisitzer nehmen die Wahl an.
Alle nciht anwesenden hben die wahl im Vorhinein angenommen ( Schrftlich)


Kassenprüfer:
Leonhrd steht zur Verfügung
Als zweiter wird Michael Schumacher vorgeschagen
Leonahrd nimmt die wahl ohne Gegenstimmen an
Michael steht zur Verfügung
Es gibt keine anderen Vorschläge oder Gegenstimmen
Michael nimmt die wahl an.

\section*{Persönlich Anträge}

Keine persönlichen Anträge


\newpage

Protokoll: Christian Kirfel


\signature{Yannick Hansen}{Vorsitzender}\hfill\signature{Christian Kirfel}{Schriftführer}


\end{document}

