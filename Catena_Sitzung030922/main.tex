\documentclass[a4paper, 11pt]{article}
\usepackage{comment} % enables the use of multi-line comments (\ifx \fi) 
\usepackage{fullpage} % changes the margin
\usepackage[german]{babel}
\usepackage[utf8]{inputenc}
\usepackage{graphicx}
\usepackage{multicol}
\usepackage{float}
\usepackage{fancyhdr}
\usepackage{enumitem}
\pagestyle{fancy} 
\usepackage{pdfpages}
%\usepackage[head=128pt]{geometry}
\title{Sitzungs-protokoll}
\author{Christian Kirfel}
\usepackage[margin=1in]{geometry}
\setlength{\footskip}{0.1pt}
\setlength{\headheight}{80pt}
\setlength{\topmargin}{0pt}
\setlength\parindent{0pt}
\fancypagestyle{style1}{
\rhead{\includegraphics[width=4cm]{logo_catena}}
\cfoot{
\makebox{}\\
\makebox{}\\
\hspace{15cm}
\makebox[0.1\linewidth]{\rule{0.1\linewidth}{0.1pt}} \hspace{1cm} \makebox[0.1\linewidth]{\rule{0.1\linewidth}{0.1pt}} \hspace{1cm} \makebox[0.1\linewidth]{\rule{0.1\linewidth}{0.1pt}} \hspace{1cm} \makebox[0.1\linewidth]{\rule{0.1\linewidth}{0.1pt}} \hspace{1cm}\\}
}

\newcommand\signature[2]{% Name; Department
\noindent\begin{minipage}{5cm}
    \noindent\vspace{3cm}\par
    \noindent\rule{5cm}{1pt}\par
    \noindent\textbf{#1}\par
    \noindent#2%
\end{minipage}}



\begin{document}
\pagestyle{style1}

\textbf{Datum: 03.09.2022} % inserir data aqui
\textbf{Sitzungszeit: 11:03 - 12:00}
\textbf{Ort: Remote, Zoom} % Definir local da reunião 

\textbf{Teilnehmer:} %
\begin{description}
\item Yannick Hansen, Vorsitzender
\item Kimberly Sarlette, stellv. Vorsitzender
\item Volker Nießen, Geschäftsführer
\item Sebastian Heinen, Schatzmeister
\item Christian Kirfel, Schriftführer
\item Dominik Hoven, Beisitzer
\item Phlipp Lanio, Beisitzer
\item Peter Engels
\item Friedhelm Schmitt
\item Michael Schumacher
\item Thomas Frauenkron
\item Leonhard Paulig
\item Alfred Feuerborn
\item Pater Hermann Preußner
\item Marcus Otte
\item Ute Stolz
\item Hannah Gillessen
\item Clara Keutgen
\item Miriam Keutgen
\item Jörg Zwitters
\item Ulrich Schloemer
\item Elke Pickartz
\end{description}

\newpage

\makebox[\linewidth]{\rule{\linewidth}{0.4pt}}\\
\textbf{Programmpunkte:} 
\begin{enumerate}
\item Begrüßung
\item Feststellung der Beschlussfähigkeit
\item Genehmigung des Protokolls der MV 2021
\item Bericht Schule
\item Bericht Klosterstiftung
\item Bericht zur Berufsberatung
\item Bericht zur Segel AG
\item Bericht Schülervertretung
\item Bericht des Vorsitzenden
\item Bericht des Schatzmeisters
\item Bericht des Kassenprüfers
\item Entlastung des Vorstandes
\item Verschiedenes
\item Persönliche Anträge
\end{enumerate}
\makebox[\linewidth]{\rule{\linewidth}{0.4pt}}\\

\newpage

\section*{Begrüßung}

Die Hauptversammlung findet wegen Covid 19 hybrid statt. Der analoge Teil sitzt in Steinfeld.
17 Teilnehmer sind anwesend, 23 sind online zugeschaltet, ein Teilnehmer stößt später hinzu.
Yannick Hansen erinnert an die im Anschluss stattfindende Veranstaltung zur Feier des 41-jährigen Vereinsjubiläums.



\section*{Beschlussfähigkeit}

Die Beschlussfähigkeit wird auf Antrag des Vorsitzenden festgestellt.

\section*{Genehmigung des Protokolls der MV 2021}

Das Protokoll wird ohne Widerspruch genehmigt.

\section*{Bericht Schule}

Jörg Zwitters berichtet über das Schulleben im vergangenen Jahr.
Die neue Sportanlage wird von Träger, Catena und Stiftung finanziert.
Herr Ohlerth ist in Pension
Herr Schmitz verlässt die Schule aufgrund der Entfernung.
Die Segel AG ist wegen Corona noch auf Eis gelegt. Die Amadeus wurde renoviert.
Herr Frauenrath ist in Pension.
Im Moment sind 730 Schüler auf der Schule. Die Schule ist dreizügig. 50 Lehrer sind angestellt.
Ukrainische Schüler sind in den Klassen 7,8 und 9.
2024 fahren alle Schüler nach Rom, anlässlich 100 Jahre Salvatorianer.

\section*{Bericht Stiftung}

Michael Schumacher berichtet.
Die Stiftung besteht seit 2004 und hat ein Kapital von zwei Millionen Euro.
Gegründet wurde die Stiftung durch die Salvatorianer unter Zustiftung von Gläubigen.
Steinfeld wird als kulturelles Zentrum unterstützt.
Die Verantwortung für das Hotel liegt bei Herrn Scheidweiler, für die Basilika beim Bistum und für die Schule bei den Salvatorianern.
Im Rahmen der Digitalisierung 2016-2018 förderte die Stiftung die Schule mit 400.000 Euro.
Im Rahmen von Klang von Steinfeld finden drei Konzerte pro Jahr statt. Die Stiftung fördert dies mit 5000 Euro.
Für die Opfer der Flut 2021 wurden 40.000 Euro gespendet.
In den kommenden Jahren sind 400.000 Euro Spenden für den Schulhof und 300.000 Euro Spenden für die Schulträgerschaft geplant.
Mit dem Ziel des langfristigen Erhalts von Kloster und Schule ist das Projekt Stiftung 2.0 neu gedacht, neu gemacht begonnen.
Die Ziele sind Fundraising, das Zusammenbringen aller Akteure und ein gemeinsamer Kalender.

\section*{Bericht zur Berufsberatung}

Yannick Hansen berichtet.
Informationen zu Work and Travel, Ausbildung und Studium wurden bereitgestellt.
Das Buddy Programm ist erfolgreich und kommt gut an.
So wird die Vernetzung von Catena und Schülern verbessert.

\section*{Bericht Segel AG}

Im Moment gibt es keine Segel AG.
Die Segel AG soll nächstes Jahr wieder beginnen.


\section*{Schülervertretung}

Kimberly Sarlette übernahm die Kommunikation mit der SV im letzten Jahr.
Es gibt einen neuen SV Raum.
Die SV nimmt am Zukunftstag teil.


\section*{Bericht des Vorsitzenden}


Yannick Hansen berichtet.

Eine virtuelle Tour der Schule wurde mit 360 Grad Kamera gedreht.
MeinVerein ist das neue Mitgliederverwaltungsportal.
Das Portal ist dezentral, selbstwervaltet und hat einen internen Chat.

Für das Jahr 2023 sind mehrere Workshops geplant.
Es soll auch 2023 ein Schul und Catena Fest geben.
Der SV Ausflug wird geplant.
Die Segel AG soll wieder beginnen.
Das Projekt Berufsberatung soll auch 2023 unterstützt werden.

Die Mitgliederentwicklung ist konstant.
Unter den jüngeren ist Zuwachs zu verzeichnen.
Karteileichen wurden entfernt.
Es wurde eine Task Force zur Mitgliedergewinnung geplant.


\section*{Bericht des Schatzmeisters}

Der Kontostand ist wegen Corona noch hoch, wird aber abgebaut.
Die Einnahmen bestehen aus den Mitgliedsbeiträgen sowie den Spenden für Flutopfer und Segel AG.
Ausgaben wurden für die Calisthenics Anlage, die Schulträgerschaft, Flutopferhilfe und die Jubiläumsfeier getätigt.


\section*{Bericht des Kassenprüfers}

Die Kassenführung war in Ordnung.
Die Entlastung des Schatzmeisters erfolgt.


\section*{Entlastung des Vorstandes}

Es gibt keine Gegenstimmen zur Entlastung des Vorstandes. Damit wird der Vorstand mit Bestätigung des Protokolls entlastet.

\section*{Wahl des Vorstands}

Eine Wahl des Vorstands findet 2023 wieder statt.


\section*{Verschiedenes}

\section*{Persönliche Anträge}

\newpage

Protokoll: Leonhard Paulig, Christian Kirfel


\signature{Yannick Hansen}{Vorsitzender}\hfill\signature{Christian Kirfel}{Schriftführer}

\end{document}
