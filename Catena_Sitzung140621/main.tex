\documentclass[a4paper, 11pt]{article}
\usepackage{comment} % enables the use of multi-line comments (\ifx \fi) 
\usepackage{fullpage} % changes the margin
\usepackage[german]{babel}
\usepackage[utf8]{inputenc}
\usepackage{graphicx}
\usepackage{multicol}
\usepackage{float}
\usepackage{fancyhdr}
\usepackage{enumitem}
\pagestyle{fancy} 
\usepackage{pdfpages}
%\usepackage[head=128pt]{geometry}
\title{Sitzungs-protokoll}
\author{Christian Kirfel}
\usepackage[margin=1in]{geometry}
\setlength{\footskip}{0.1pt}
\setlength{\headheight}{80pt}
\setlength{\topmargin}{0pt}
\setlength\parindent{0pt}
\fancypagestyle{style1}{
\rhead{\includegraphics[width=4cm]{logo_catena}}
\cfoot{
\makebox{}\\
\makebox{}\\
\hspace{15cm}
\makebox[0.1\linewidth]{\rule{0.1\linewidth}{0.1pt}} \hspace{1cm} \makebox[0.1\linewidth]{\rule{0.1\linewidth}{0.1pt}} \hspace{1cm} \makebox[0.1\linewidth]{\rule{0.1\linewidth}{0.1pt}} \hspace{1cm} \makebox[0.1\linewidth]{\rule{0.1\linewidth}{0.1pt}} \hspace{1cm}\\}
}

\newcommand\signature[2]{% Name; Department
\noindent\begin{minipage}{5cm}
    \noindent\vspace{3cm}\par
    \noindent\rule{5cm}{1pt}\par
    \noindent\textbf{#1}\par
    \noindent#2%
\end{minipage}}



\begin{document}
\pagestyle{style1}

\textbf{Datum: 14.06.2021} % inserir data aqui
\textbf{Sitzungszeit: 19:00 - 00:00}
\textbf{Ort: Remote, Zoom} % Definir local da reunião 

\textbf{Teilnehmer:} %
\begin{description}
\item Yannick Hansen, Vorsitzender
\item Dr.~Michael Schumacher, stellv. Vorsitzender
\item Volker Nießen, Geschäftsführer
\item Friedhelm Schmitt, Schatzmeister
\item Christian Kirfel, Schriftführer
\item Jörg Zwitters, Verbindungslehrer
\item Sebastian Heinen Beisitzer Kasse
\item Dominik Hoven, Beisitzer
\item Max Frauenrath, Beisitzer
\end{description}

\makebox[\linewidth]{\rule{\linewidth}{0.4pt}}\\
\textbf{Programmpunkte:} 
\begin{enumerate}
\item Mitgliederverwaltung
\item Weiteres
\end{enumerate}
\makebox[\linewidth]{\rule{\linewidth}{0.4pt}}\\

\newpage

\section*{Mitgliederverwaltung}

Volker hat die Beiträge ins neue System eingepflegt.
Laut Program sind die Sepa Mandate angelaufen. Sie behalten nur 3 Jahre ihre Gültigkeit.
Es ist unklar, ob dies wirklich zu Problemen führt.
Vermutlich muss man bei der Bank oder beim Bereitsteller der website nachfragen.

Yannick hat gegoogled.
Grundsätzlich gilt hier unbefristet. Außer es wird 3 Jahre nicht genutzt.
Mitgliedsnummern wurden zuvor von der alten Software vergeben.
Es muss geprüft werden, ob die Nummern gleich geblieben sind.

Im Import fehlt noch der alte Kassenstand?
Außerdem ist die Frage ob fehlende Daten problematisch sind.

Bei einige liegen ungenaue Daten vor. Die alte Liste muss kontrolliert werden.
Das neue Program hat keine Geburtsnamen eingepflegt.
Das sollte als Option möglich werden.

Viele Mitglieder zahlen noch den alten Beitrag, wurden aber auch auf diese Art eingepflegt.
Manche der eingezogenen Beträge sind nicht verstanden. Das mag darin liegen, dass sich Beitrag und Spende addieren.

Studentenbeitrag:
Aktuell wurde der Studentenbeitrag eingezogen. Eigentlich müsste der nun geändert werden.

Änderung der summe sollte mit dem Geburtsdatum passieren, nicht mit dem Geschäftsjahr.
Auch hier stellt sich die Frage, ob das zu asutritten führt.

Bei einigen jungen Mitgliedern fehlen Bankdaten. Dies liegt daran, dass sie vorher im Schnupperjahr waren.

\section*{Sommerfest}

Bisher sprechen die Maßnahmen dagegen.
Jörg Zwitters verspricht neue Infos zu Beginn des Schuljahres zu kommunizieren

\section*{Ausgaben}

In Steinfeld wurde in die Sportanlagen investiert.
Jörg Zwitters kommuniziert, dass der Bau gute Fortschritte macht.

\section*{Segel AG}

Kann nicht stattfinden. Ende des Jahres muss hier wieder investiert werden.

\section*{Vorbereitung Studienzeit}

Dominik übernimmt das duale Studium, Kim übernimmt das Auslandssemester, Martin das reguläre Studium
Angeboten werden sollte diese Veranstaltung später auch im Herbst, um für Ausbildung und duales Studium die richtigen deadlines zu treffen.

\section*{Ansprechpartner}

Kims Plan
Bessere Einbindung

Hier muss ein Gesamtkonzept gefunden werden und eventuell ist der Weg über Lehrerdirektanfrage der richtigen

Man könnte Vorträge halten, statt Sommerfest



\newpage

Protokoll: Christian Kirfel


%\signature{Yannick Hansen}{Vorsitzender}\hfill\signature{Christian Kirfel}{Schriftführer}


\end{document}
