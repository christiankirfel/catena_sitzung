\documentclass[a4paper, 11pt]{article}
\usepackage{comment} % enables the use of multi-line comments (\ifx \fi) 
\usepackage{fullpage} % changes the margin
\usepackage[german]{babel}
\usepackage[utf8]{inputenc}
\usepackage{graphicx}
\usepackage{multicol}
\usepackage{float}
\usepackage{fancyhdr}
\usepackage{csquotes}
\usepackage{enumitem}
\pagestyle{fancy} 
\usepackage{pdfpages}
%\usepackage[head=128pt]{geometry}
\title{Sitzungs-protokoll}
\author{Christian Kirfel}
\usepackage[margin=1in]{geometry}
\setlength{\footskip}{0.1pt}
\setlength{\headheight}{80pt}
\setlength{\topmargin}{0pt}
\setlength\parindent{0pt}
\fancypagestyle{style1}{
\rhead{\includegraphics[width=4cm]{logo_catena}}
\cfoot{
\makebox{}\\
\makebox{}\\
\hspace{15cm}
\makebox[0.1\linewidth]{\rule{0.1\linewidth}{0.1pt}} \hspace{1cm} \makebox[0.1\linewidth]{\rule{0.1\linewidth}{0.1pt}} \hspace{1cm} \makebox[0.1\linewidth]{\rule{0.1\linewidth}{0.1pt}} \hspace{1cm} \makebox[0.1\linewidth]{\rule{0.1\linewidth}{0.1pt}} \hspace{1cm}\\}
}

\newcommand\signature[2]{% Name; Department
\noindent\begin{minipage}{5cm}
    \noindent\vspace{3cm}\par
    \noindent\rule{5cm}{1pt}\par
    \noindent\textbf{#1}\par
    \noindent#2%
\end{minipage}}



\begin{document}
\pagestyle{style1}

\textbf{Datum: 02.09.2022} % inserir data aqui
\textbf{Sitzungszeit: 11:03 - 12:00}
\textbf{Ort: Steinfeld Bioraum, Zoom} % Definir local da reunião 

\textbf{Teilnehmer:} %
\begin{description}
\item Yannick Hansen, Vorsitzender
\item Kimberly Sarlette, stellv. Vorsitzender
\item Volker Nießen, Geschäftsführer
\item Sebastian Heinen, Schatzmeister
\item Christian Kirfel, Schriftführer
\item Dominik Hoven, Beisitzer
\item Phlipp Lanio, Beisitzer
\item Peter Engels
\item Friedhelm Schmitt
\item Michael Schumacher
\item Thomas Frauenkron
\item Leonhard Paulig
\item Alfred Feuerborn
\item Pater Hermann Preußner
\item Marcus Otte
\item Ute Stolz
\item Hannah Gillessen
\item Clara Keutgen
\item Miriam Keutgen
\item Jörg Zwitters
\item Ulrich Schloemer
\item Elke Pickartz
\end{description}

\newpage

\makebox[\linewidth]{\rule{\linewidth}{0.4pt}}\\
\textbf{Programmpunkte:} 
\begin{enumerate}
\item Begrüßung
\item Feststellung der Beschlussfähigkeit
\item Genehmigung des Protokolls der MV 2021
\item Bericht Schule
\item Bericht Klosterstiftung
\item Bericht zur Berufsberatung
\item Bericht zur Segel AG
\item Bericht Schülervertretung
\item Bericht des Vorsitzenden
\item Bericht des Schatzmeisters
\item Bericht des Kassenprüfers
\item Entlastung des Vorstandes
\item Verschiedenes
\item Persönliche Anträge
\end{enumerate}
\makebox[\linewidth]{\rule{\linewidth}{0.4pt}}\\

\newpage

\section*{Begrüßung}

Die Hauptversammlung findet hybrid statt. Der analoge Teil sitzt in Steinfeld im Bioraum.
Alle sind gut in den Start getagt.

Auf das in Folge stattfindenende Schul- und Catenafest wird eingeladen.




17 Teilnehmer sind anwesend, 23 sind online zugeschaltet, ein Teilnehmer stößt später hinzu.
Yannick Hansen erinnert an die im Anschluss stattfindende Veranstaltung zur Feier des 41-jährigen Vereinsjubiläums.



\section*{Beschlussfähigkeit}

Die Beschlussfähigkeit wird auf Antrag des Vorsitzenden festgestellt.

\section*{Genehmigung des Protokolls der MV 2022}

Das Protokoll wird ohne Widerspruch genehmigt.

\section*{Bericht Schule}

Joerg Zwitters berichtet. Das HJK waechst weiterhin. Drei volle Klassen, 28 Schueler jeweils.
Heiner Schmidt und Willy Frauerath und Sabine Flesch Kommander wurden im Winter verabschiedet.
Das Kollegiat verjuengt sich was eine erhoehte Dynamik zur Folge hat.
Durch Regen und Flugschaden und der erhoehten Schuelerzahl durch G8 und G9 Wechsel wurde der Musiktrakt umgestaltet.
Dadurch wurden mehr Klassenzimmer hinzugefuegt, Bauarbeiten sind noch im Gange.
Dadurch sollte auch der Schimmelbefall eingedaemmt werden.
Auf den Catena Tag und das Outdoor Klassenzimmer wird spaeter eingegangen.
Die Unterstuetzung der SV Arbeit ist sehr wichtig.

Die Segel AG lief ueblicherweise ueber Mundpropaganda. Durch Corona enstand eine Luecke.
Im letzten Kurs erhielt nur eine Schuelerin den Schein.
Dieses Jahr haben drei von drei den Kurs bestanden. Die Begeisterung haelt sich in Grenzen.
\enquote{HJK waechst und gedeiht}

\section*{Bericht der Stiftung}

Martin stellt Michael, sich und die Aufgaben der Stiftung vor.
DIe Bekanntheit der Stiftung in der Gegend soll vergroessert werden um die Mittel der Stiftung zu vergroessern.
Einige der Mittel sind als Stiftungkapital gefroren. Gefoerdert werden unter anderem eine Konzertreihe.
In diesem Rahmen ist eine ukrainische Chroleitering hier, die mit einem Projektchor aus der Gegend singt.
Der erwaehnte erneurte Klassenraum wurde mitfinanziert.

Maria Wald wurde ebenfalls von Herrn Scheitweiler renoviert.
Dort folgen Restaurant und eiegnes Bier.
Auch ein Gaestehaus mit eher besinnlichem Teil soll dort folgen.

Die geringe Zahl der Salvatorianer sind weiter sehr egnagiert.
Michael hat ncihts hinzuzufuegen.

\section*{Bericht Berufsberatung}

Yannick berichtet.
Praesentationen finden zweimal im Jahr statt.
Hir wird Fokus auf Studium, Ausbidlung und \textit{Work and Travel} gelegt.
Die Resonanz der Schueler ist aktiv und engagiert.

Yannick betont, dass er diese Arbeit als Ziel der Catena sieht.
Auch zum Thema Stipendium wird gesprochen.

\section*{Catena Segel AG}

Es soll auch ein Segelkurs fuer Catena Mitglieder ermoeglicht werden.


\section*{Bericht Schuelervertretung}

Kimberley berichtet.
Es gab eine selbstorganisierte Fahrt nach Bonn mit den Klassensprechern.
Dort wurden kleine direkt in der Schule umsetzbare Projekte wie ein Schulkino geplant.
Ausserdem wurde gemeinsam mit der SV der SV Raum neu getsaltet.
Die Rechnung wurde seitens der Catena getragen.
Der Raum wird aufgrund des Umbaus noch als Abstellraum genutzt.

\section*{Bericht zum Catena Tag} oder Vorstitz

Yannick berichet.
Ziele des erstmalig am letzten Freitag stttgefundenen Catena Tags ist das den Schuelern der Q2
ein Tag mit Nutzen aber auch Unterhaltung geboten wird.
Dort werden verschiedene Workshops gemacht. Bewerbungstraining, Lampenfieber, Stressbewaeltigung, soziale Projekte
Nachhaltiger Umgang mit Ressourcen
Ein Deeskalationstraining musste leider verschobeen werden bleibt aber Teil der Planung.
Am Ende gab es Pizza
Der Tag soll im folgenden Jahr erneut statfinden und eine staerkere Einbindung von Catena Mitgliedern ist angestrebt.

Am Catena Tag wie auch heute gab es einen Eiswagen. Das Feedback war ausgesprochen positiv.
Der Wagen war preiswert und kam von der Eisdiele Blankenheim.

Eine Catena Fahrt nach Muenchen findet vom 6-8 Oktober statt.
Geplant ist eine gemeiensame Anreise mit der Bahn. Eine Uebernachtung bei den Salvatorianern ist nicht moeglich
Pater Hubertus(?) begleitet beim Sightseeing durch die Stadt und ein Fruehstueck mit den Salvatorianern findet statt.
Bildungsur;laub Muenchen. Hier wird ein kultureller Aspekt gefoerdert.

Auch im naechsten Jahr soll eine solche Fahrt mit kulturellem Mehrwert stattfinden.


Fuer die Sportstaetten wurden neue Tischtennisplatten seitens der Catena und des Foerderveins unterstuetzt. finanziert.

Es gibt nun ein Catena Buero gegenueber der Orgel.
Der Raum kann gleich besichtigt werden.
Dominik betont, dass der Raum durch den Vorstand selbst renoviert wurde.


\section*{Ausblick auf naechstes Jahr 2024} weiter bericht des Vorsitzenden

100. jaehriges Jubilaeum mit Schul und Catena Fest am 7.9.24
Outdoor Klassenzimmer soll seitens der Catena unter diesem Anlass geplant werden.
Als Platz dafuer bietet sich der bisher ungenutzte Platz zwischen Aula und Korridor an.

Damit ausreichend Zimmer zum Fest geblockt werden koennen, muesste man bis Maerz Zusagen haben. (Volker)

Ausserdem findet anlaesslich des Jubilaeums eine Romfahrt fuer die ganze Schule statt.
Zur Dokuentation faehrt ein Teil des Vorstandes mit.
Die Idee beruht auf einer Erfahrung die Thomas Frauenkron auf einer ehemaligen Romfaht gemacht hat.
Dort hat er ein aehnliches Projekt einer anderen Schuel gesehen.

Yannik wirbt dafuer, dass wir in beratender Taetigkeit auch gerne andere Catena Mitgleider einbeziehen wuerden.
Eine solche Taetigkeit kann fuer Schuler oder Mitglieder statfinden.



\section*{Entgegennahme der Vorstandsberichte}



\section*{Bericht des Schatzmeisters}

Zwecks Dokumentation werden die gezeigten Folien eingebunden.

Die Einnahmen bestehen fast nur aus Mitgliedsbeitraegen im vorigen Jahr.
Zwecks Segel AFG und Flutspenden kamen einmalig Spenden dazu.

Die geringeren Einnahmen lagen an technischen Problemen mit der neuen Software.


Die Ausgaben der beiden letzten Jahre werden im Vergleich als Tortendiagram dargestellt.
Es gibt keine Fragen


\section*{Bricht des Geschaeftsfuehrers}

Volker berichtet.

Der Gewinn der Juengeneren bleibt aus aber es fallen aeltere weg.

Gegeben der betriebene Aufwand zur Mitgliedergewinnung ist dies etwas traurig.

Es wird erfragt ob es die 1 Euro bis 25 Beitraege noch gibt . Gibt es.
Maritn fragt ob Lehrer werben.
Joerg kann nur fuer sich selbst sprechen.
Die Bereitschaft sich zu beteiligen ist sehr klein. Auch wenn es um Hilfe beim Fest geht.
Einige andere Kollegen werben auch.

Diese 1 Euro Beitraege gibt es erst seit  Jahren aber man muss etwas Geduld zeigen.

Auf Bachfrage wie einfach der Beitritt ist, gibt es die Moeglichkeit ueber QR COde beizutraten.

Frau Hansen berichtet, dass die Catena in ihrer Schulzeit wenig wahrgenommen wurde.
Sie bewribt, dass die Catena bei den Eltern vorstellig wird.
Michael erwaehnt, dass auch Eltern die Moeglichkeit haben sollten.

Auch am Tag der offenen Tuer muss schon Praesenz gezeigt werden.

Bei der Elternschaft herrscht die Konkurrenz mit dem Foerderverein.

Eltern bei den elternabenden erreichen.

Bezeugelich der Mitgleiderbeiotraege werden historische Zahlen erfragt.
Historaich muessen die Internatsgruendungsjahrgaenge betont werden.

Laut Martin war die Schule frueher ein staerkerer sozialer Schwerpunkt

Martin merkt an, dass andere Schule staerkere Foerdungen an ihre Ehemaligen fuer die Alumnivereien fordern.

Kim schlaegt vor email adressen zu sammeln. Frueher gab es Briefe.

Vorschlag auf den Abschlussfahren werben.

Volker schlaegt vor jeweils eine Person pro Stufe werung machen zu lassen.
sebastian sagt dass seine Stufe naechstes Jahr auf dem Schulfest ihr Treffen veranstaltet.

Wenn das Schulfest endet, gehen viele Eltern und Kinder.

Keine Trennung zwischen Schul und Catena Fest

\section*{Bericht des Kassenprüfers}

(Wer ist der?)
Die Kassenführung war in Ordnung.
Die Entlastung des Schatzmeisters erfolgt.

\section*{Entlastung des Vorstandes}

Es gibt keine Gegenstimmen zur Entlastung des Vorstandes. Damit wird der Vorstand mit Bestätigung des Protokolls entlastet.

Es wird geklopft fuer Schatzmeister, Kassenpruefer und Vorstand.

\section*{Wahl des Vorstands}

Einige der Beisitzer treten zurueck.

Yannick Hansen stellt sich zur Wahl. Es gibt keinen Gegenstimmen oder Gegenkandidaten. Er nimmt die Wahl an
Kim stellt sich zur Wahl. Es gibt keinen Gegenstimmen oder Gegenkandidaten. Er nimmt die Wahl an
Volker stellt sich zur Wahl. Es gibt keinen Gegenstimmen oder Gegenkandidaten. Er nimmt die Wahl an
Sebastian stellt sich zur Wahl. Es gibt keinen Gegenstimmen oder Gegenkandidaten. Er nimmt die Wahl an
Leonhard Paulig stellt sich zur Wahl. Es gibt keinen Gegenstimmen oder Gegenkandidaten. Er nimmt die Wahl an
Alfred Kahl stellt sich zur Wahl zum Kassenpruefer. Es gibt keinen Gegenstimmen oder Gegenkandidaten. Er nimmt die Wahl an
Christian Kirfel stellt sich zur Wahl. Es gibt keinen Gegenstimmen oder Gegenkandidaten. Er nimmt die Wahl an

Beisitzer: Joerg, Dominik, Martin, Florian, Henrik
Yannick Hansen stellt sich zur Wahl. Es gibt keinen Gegenstimmen oder Gegenkandidaten. Er nimmt die Wahl an
Yannick Hansen stellt sich zur Wahl. Es gibt keinen Gegenstimmen oder Gegenkandidaten. Er nimmt die Wahl an
Yannick Hansen stellt sich zur Wahl. Es gibt keinen Gegenstimmen oder Gegenkandidaten. Er nimmt die Wahl an
Yannick Hansen stellt sich zur Wahl. Es gibt keinen Gegenstimmen oder Gegenkandidaten. Er nimmt die Wahl an
Yannick Hansen stellt sich zur Wahl. Es gibt keinen Gegenstimmen oder Gegenkandidaten. Er nimmt die Wahl an

Beisitzer erfuellen verschiedene Aufgaben und haben Stimmrecht.
Hier sind Segel AG, Sv Betreuuung und Schul Kontakt.



Michael wird Berater.

\section*{Antrag auf Aenderung der Satzung}

Martin dokumentiert, dass wir die Mitgliedervernetzung aufgenommen wird.
Das ist schwer, weil wir dann nicht mehr gemeinnuetzig sind.
Das darf nicht im Protokoll erscheinen.
Unsere Aufgaben sind immer allgemeinnuetzig und kulturell.
Wir foerdern ein kulturelles Ereignis.

Eine Satzung wird nicht geaendert.

\section*{Verschiedenes}

\section*{Persönliche Anträge}

Hier geht es um eine Vollendung der Glaswand.
In die Glaswand wurden Schuelernamen einegtragen.
Dies widersprach den Wuenschen der Catena.
Sollte man einen Gesamtentwurf vorschlagen.
Man sollte sich hier bei den Schuelern keine Feinde machen.

\section*{Abschluss}

Einladung zum MIttagessen, Catena Buero und Schulfest.

\newpage

Protokoll: Christian Kirfel


\signature{Yannick Hansen}{Vorsitzender}\hfill\signature{Christian Kirfel}{Schriftführer}

\end{document}
